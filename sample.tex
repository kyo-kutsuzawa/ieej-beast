\documentclass[english,letter]{ieej-e}
%\documentclass[english]{ieej-e-samcon}
\usepackage{comment}
\usepackage{url}
\usepackage{hologo}

\YEAR{2019}
\title{BeaST file for IEEJ}
\authorlist{%
 \authorentry[someone@example.com]{Taro Tanaka}{s}{author1}
}
\affiliate[author1]
{Anonymous}

\begin{document}
\begin{abstract}
A BeaST file according to the IEEJ style is provided.
Here shows examples of references.
\end{abstract}
\begin{keyword}
  \LaTeX, \hologo{BibTeX}, BeaST, IEEJ
\end{keyword}
\maketitle


\section*{Citations}
\Cite{Denki2001,Knuth1994,Yamada2001}


\section{SAMCON}
Name of the author(s): ``Title'', Name of Publication, Volume, Number, Page (Year and Month of Publication)
Include all the author's names in full.
Try to avoid abbreviating the title.
from \url{http://www2.iee.or.jp/~diic/samcon/}

Place references in the right order according to the IEEJ editing style; e.g.,authors' names, initials, title of article, journal abbreviation,volume number, pages, and publication year.
Journals areitalicized as \texttt{\textbackslash itshape}, and titles of papers enclosed with `` '' in plain text.
\footnotesize
\begin{enumerate}
  \item D.E. Knuth, The TEXbook, Addison-Wesley (1994)
  \item R. Seroul \& S. Levy: A Beginner's Book of TEX, Springer-Verlag (1989)
  \item D. Salomon: The Advanced TEXbook, Springer-Verlag (1995)
  \item V. Eijkhout: TEX by Topic, Addison-Wesley (1991)
  \item P.W. Abrahams: TEX for the Impatient, Addison-Wesley (1992)
  \item S. von Bechtolsheim: TEX in Practice, Springer-Verlag (1993)
  \item G. Gr\"atzer: Math into TEX–A Simple Introduction to AMS-LATEX, Birkh\"auser(1993)
  \item N. Walsh: Making TEX Work, O'Reilly \& Associates (1994)
  \item L. Lamport, LATEX:  A Document Preparation System, Second Edition, Addison-Wesley (1994)
  \item M. Goossens, F. Mittelbach \& A. Samarin: The LATEX Companion, Addison-Wesley (1994)
  \item M. Goossens, S. Rahts, and F. Mittelbach: The LATEX Graphics Companion, Addison-Wesley (1997)
\end{enumerate}
\normalsize
from \url{http://www2.iee.or.jp/~diic/samcon/submission/IEEJ-SAMCON-template.pdf}


\section{JIA}
\begin{enumerate}
  \item D.E. Knuth, The \TeX{}book, Addison-Wesley (1994)
  \item R. Seroul \& S. Levy: A Beginner's Book of \TeX, Springer-Verlag (1989)
  \item D. Salomon: The Advanced \TeX{}book, Springer-Verlag (1995)
  \item V. Eijkhout: \TeX\ by Topic, Addison-Wesley (1991)
  \item P.W. Abrahams: \TeX\ for the Impatient, Addison-Wesley (1992)
  \item S. von Bechtolsheim: \TeX\ in Practice, Springer-Verlag (1993)
  \item G. Gr\"{a}tzer: Math into \TeX--A Simple Introduction to \hologo{AmSLaTeX}, Birkh\"{a}user (1993)
  \item N. Walsh: Making \TeX\ Work, O'Reilly \& Associates (1994)
  \item L. Lamport, \LaTeX: A Document Preparation System, Second Edition, Addison-Wesley (1994) 
  \item M. Goossens, F. Mittelbach \& A. Samarin: The \LaTeX\ Companion, Addison-Wesley (1994)
  \item M. Goossens, S. Rahts, and F. Mittelbach: The \LaTeX\ Graphics Companion, Addison-Wesley (1997)
\end{enumerate}
from \url{http://www2.iee.or.jp/ver2/ias/IEEJ-JIA/}


\section{電気学会論文誌}
引用文献の表記は原則として英文とする。ただし,英文表記のない文献を引用する場合は日本語でも差し支えない。 著者名は著者全員を,またタイトルは省略しないで記載してください。 記載内容は,『著者名:「題目」,書名,巻,号,ページ(西暦発行年月)』です。下記をご参照下さい。
\begin{enumerate}
  \item 日本語論文などの場合(英語・日本語の併記,(注)ただしタイトルなどに英文表記がない場合は日本語のみとする。\\
  T. Denki, M. Hanai, and G. Misaki: ``Future Technology for Power System Analysis'', IEEJ Trans.PE, Vol.130, No.1, pp.130-136 (1999-1) (in Japanese)\\
  電気太郎・花井桃子・岬 五郎:「電力系統解析技術の将来」,電学論B,Vol.130, No.1, pp.130-136(1999-1)
  \item 国際会議などの論文集の場合\\
  B. Yamada: ``Experimental studies of new micromechanical vibration systems'', Proc. IEEE Conf. on Micro-mechanical Component, No.21, pp.123-145, Paris, France (1999-4)
  \item 単行本などの場合(ただし,日本語単行本の表記は上記aの(注)のような取扱いとする。)\\
  Y. Sankar: Management of Technological Change, p.10, John Wiley, New York (1991)
\end{enumerate}
from \url{http://www.iee.jp/wp-content/uploads/honbu/32-doc-kenq/t-tebiki.pdf}


\bibliographystyle{samcon}
\bibliography{references}


\end{document}
